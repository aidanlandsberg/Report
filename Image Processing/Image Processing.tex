\documentclass[11pt, oneside]{article}   	% use "amsart" instead of "article" for AMSLaTeX format
\usepackage{geometry}                		% See geometry.pdf to learn the layout options. There are lots.
\geometry{letterpaper}                   		% ... or a4paper or a5paper or ... 
%\geometry{landscape}                		% Activate for for rotated page geometry
%\usepackage[parfill]{parskip}    		% Activate to begin paragraphs with an empty line rather than an indent
\usepackage{graphicx}				% Use pdf, png, jpg, or eps� with pdflatex; use eps in DVI mode
\usepackage{natbib}								% TeX will automatically convert eps --> pdf in pdflatex		
\usepackage{amssymb}

\title{\Large{Week 2: Camera Theory}}
\author{Aidan Landsberg}
\date{\today}							% Activate to display a given date or no date

\begin{document}
\maketitle
%\newpage
\section{Introduction}
In the MonoSLAM approach presented by Davison, the primary sensor information is gathered from a \textbf{single} low-cost digital camera. The following section, first and foremost, aims to present a mathematical and geometrical approach to the modelling of the aforementioned camera in its ideal form. Thereafter, the ideal model can be modified to incorporate the effects of non-idealities such as distortion. This section will also provide a sufficient approach to reconstruct an estimate of the original three dimensional position form a sequence of two dimensional data contained within the cameras images.   
\subsection{The Pinhole Camera Model}
This classical model provides a reasonable approximation of a three-dimensional point in world and approximates this position according to a 2-dimensional plane. Though this model is widely utilised in the computer vision field, it is important to note that the model itself is an approximation that incorporates a number of assumptions. The models convenience as well as its simplicity make it a popular choice within practice. The model though, can be adapted to limit certain assumptions and better approximate the properties that coincide with available digital cameras.
\subsubsection{Description of the Model}
The pinhole camera model can be described as a two-dimensional plane, containing the projections of the three dimensional coordinate. The process of representing a 3D coordinate in terms of a 2D coordinate is known as \textit{perspective projection}. The two-dimensional plane is commonly referred to as the \textit{pinhole plane} which contains a infinitesimal hole at its centre - the \textit{pinhole}. The camera possess its own 3D coordinate system with coordinate axes $X_C$, $Y_C$ and $Z_C$ (also referred to as the cameras \textit{optical axis}). The pinhole is situated at the origin of this 3D coordinate system - this is also referred to as the \textit{optical centre}, \textit{O}. The image plane is located at a positive distance \textit{f} from the optical centre \textit{O} along the negative $Z_C$-axis and is parallel to the pinhole plane.      
\begin{figure}[htbp]
\begin{center}

\caption{default}
\label{default}
\end{center}
\end{figure}

\end{document}  